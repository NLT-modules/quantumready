\hypertarget{biomimetica-en-quantum-technologie}{%
\section{Biomimetica en Quantum
Technologie}\label{biomimetica-en-quantum-technologie}}

Hier moet een introtekstje voor de startopdracht.

\texttt{\{exercise\}\ Startopdracht\ In\ dit\ hoofdstuk\ maak\ je\ kennis\ met\ de\ twee\ concepten\ die\ de\ basis\ van\ deze\ module\ vormen:\ biomimetica\ en\ quantum\ sensoren.}

\hypertarget{wat-is-biomimetica}{%
\subsection{Wat is biomimetica?}\label{wat-is-biomimetica}}

Goed kijken naar hoe de natuur werkt kan ons helpen met het ontwikkelen
van nieuwe technologie. Veel van onze technologische uitdagingen zijn
namelijk al door de natuur opgelost in een proces van miljarden jaren
aan evolutie. Hoe houd je overdag iets koel, of 's nachts warm? Hoe zet
je licht om in bruikbare energie? Hoe maak je iets waterafstotend? Dat
zijn allemaal vragen waar wetenschappers, technici en ingenieurs zich
mee bezig houden. Soms kijken ze daarvoor goed naar de natuur om nieuwe
oplossingen te vinden. Bewust de functionaliteit van de natuur nadoen
heet biomimetica of in het Engels biomimicry. In plaats van leren over
de natuur leer je van de natuur. Het volgende voorbeeld laat dat goed
zien.

\hypertarget{vliegen-als-voorbeeld-van-biomimetica}{%
\subsection{Vliegen als voorbeeld van
biomimetica}\label{vliegen-als-voorbeeld-van-biomimetica}}

In een mythe uit de Griekse oudheid maakt de uitvinder Daedalus vleugels
uit veren en bijenwas voor hem en zijn zoon om van het eiland Kreta weg
te kunnen vliegen. Ondanks waarschuwingen van zijn vader vliegt Icarus
te dicht bij de zon waardoor de bijenwas smelt. Icarus stort neer in de
Icarische Zee en verdrinkt \href{fig1.1}{figuur 1.1a}.

Het tragische verhaal van Icarus illustreert dat mensen al duizenden
jaren fantaseren over zelf kunnen vliegen. Het duikt bijvoorbeeld ook op
in het werk van Leonardo Da Vinci omstreeks 1480 Hij bestudeert vogels
en ontwerpt meerdere kunstvleugels \href{fig1.1}{figuur 1.1b}, inclusief
een harnas om de vleugels om te doen.

In 1903 voeren de gebroeders Wright de eerste gemotoriseerde
vliegtuigvlucht uit \href{fig1.1}{figuur 1.1c}. Ze hadden in de jaren
daarvoor uitgebreid gewerkt aan het ontwerpen van vleugels die tijdens
de vlucht van vorm kunnen veranderen om zo het toestel stabiel te
houden, net zoals vogels dat doen. Mocht je ooit in een vliegtuig bij
het raam zitten let dan eens op de vleugel. Die verandert tijdens de
vlucht ook van vorm.

(fig1.1)=

\begin{verbatim}
<figure style="margin-right: 20px;">
    <img src="https://cdn.mathpix.com/cropped/2024_06_24_0f091e854944f2a35adag-03.jpg?height=308&width=317&top_left_y=1896&top_left_x=184" alt="Figuur 1.1a: De val van Icarus, geschilderd door Jacob Peter Gowy (17e eeuw)" style="height: 200px;">
    <figcaption>Figuur 1.1a: De val van Icarus, geschilderd door Jacob Peter Gowy (17e eeuw) </figcaption>
</figure>
<figure style="margin-right: 20px;">
    <img src="https://cdn.mathpix.com/cropped/2024_06_24_0f091e854944f2a35adag-03.jpg?height=311&width=397&top_left_y=1895&top_left_x=518" alt="Figuur 1.1b: Vleugelontwerp door Leonardo da Vinci." style="height: 200px;">
    <figcaption>Figuur 1.1b: Vleugelontwerp door Leonardo da Vinci.</figcaption>
</figure>
<figure>
    <img src="https://cdn.mathpix.com/cropped/2024_06_24_0f091e854944f2a35adag-03.jpg?height=317&width=434&top_left_y=1892&top_left_x=934" alt="Figuur 1.1c: De succesvolle vliegtuigvlucht van de gebroeders Wright in 1903." style="height: 200px;">
    <figcaption>Figuur 1.1c: De succesvolle vliegtuigvlucht van de gebroeders Wright in 1903.</figcaption>
</figure>
\end{verbatim}

```\{exercise\} Nog Een Betere Vleugel Rond 1970 stelt Richard Whitcomb,
ingenieur bij NASA, voor het ontwerp van een vliegtuigvleugel aan te
passen zodat deze meer lijkt op de vleugel van een zwevende adelaar. Zie
\href{fig1.2}{figuur 1.2 a t/m c}.

(fig1.2)=

\begin{verbatim}
<figure style="margin-right: 20px;">
    <img src="https://cdn.mathpix.com/cropped/2024_06_24_0f091e854944f2a35adag-03.jpg?height=308&width=317&top_left_y=1896&top_left_x=184" alt="vleugel zonder aanpassingen" style="height: 200px;">
    <figcaption>Figuur 1.2a: Vliegtuigvleugel zonder aanpassing.</figcaption>
</figure>
<figure style="margin-right: 20px;">
    <img src="https://cdn.mathpix.com/cropped/2024_06_24_0f091e854944f2a35adag-05.jpg?height=320&width=437&top_left_y=688&top_left_x=570" alt="adelaar" style="height: 200px;">
    <figcaption> 1.2b: Zwevende adelaar. (Foto door Derek Keats).</figcaption>
</figure>
<figure>
    <img src="https://cdn.mathpix.com/cropped/2024_06_24_0f091e854944f2a35adag-05.jpg?height=317&width=368&top_left_y=687&top_left_x=998" alt="Whitcombs." style="height: 200px;">
    <figcaption>Figuur 1.2c: Uiteinde van een vliegtuigvleugel met Whitcombs aanpassing.</figcaption>
</figure>
\end{verbatim}

Zoek informatie in ten minste drie verschillende bronnen om in je eigen
woorden onderstaande vragen te beantwoorden. Geeft daarbij duidelijk aan
welke informatie uit welke bron afkomstig is. Gebruik tussen de 250 en
500 woorden.

\begin{enumerate}
\def\labelenumi{\arabic{enumi}.}
\tightlist
\item
  Wat heeft Whitcomb aan de vleugel van een vliegtuig aangepast?
\item
  Voor welke verbetering(en) heeft deze verandering gezorgd?
\item
  Welke impact heeft dat gehad op de moderne luchtvaart?
\end{enumerate}

\begin{verbatim}

(ex1.3)=
```{exercise} Match de natuur en de toepassing
Probeer de puzzle hieronder op te lossen. Links vind je 9 boxen met daarin een omschrijving van technologische toepassing geïnspireerd op een dier/plant, rechts vindt je 9 verschillende dieren/planten die bij een van deze omschrijving hoort. Probeer de goede plant/dier bij de goede omschrijving te vinden. 
Een voorbeeld: De wateropslag voor hergebruik in energiecentrales is geinspireerd door de bij.
<iframe src="https://tudelft.h5p.com/content/1292307606531729577/embed" width="1088" height="637" frameborder="0" allowfullscreen="allowfullscreen" allow="autoplay *; geolocation *; microphone *; camera *; midi *; encrypted-media *"></iframe><script src="https://tudelft.h5p.com/js/h5p-resizer.js" charset="UTF-8"></script>
\end{verbatim}

```\{exercise\} Match de natuur en de toepassing (vervolg)

In \href{ex1.3}{Opdracht 1.3} heb je kennis gemaakt met een aantal
voorbeelden van biomimetica. Bij ieder voorbeeld is een video
beschikbaar die je van de docent krijgt.

\begin{enumerate}
\def\labelenumi{\alph{enumi})}
\item
  Kies één voorbeeld van biomimetica uit \href{ex1.3}{Opdracht 1.3}.
\item
  Zoek bij dat voorbeeld, naast de video twee aanvullende bronnen.
\item
  Gebruik je bronnen om in je eigen woorden uit te leggen wat het
  verband is tussen de natuur en de technologische toepassing.
\end{enumerate}

Geef duidelijk aan welke informatie uit welke bron afkomstig is en
gebruik tussen de 250 en 500 woorden.

\begin{verbatim}

### Technologie en biomimetica

Biomimetica is niet alleen kijken naar de natuur. Het is een proces waarbij observeren, meten, onderzoeken en ontwerpen een belangrijke rol spelen. Zo vergeleek Richard Whitcomb de vleuge van een vliegtuig met de vleugel van een adelaar en haalde daar inspiratie uit. Hij heeft vervolgens verschillende modelvleugels gemaakt en de luchtweerstand gemeten in een windtunnel.

Waarom de cyphochilus kever uit [Opdracht 1.3](ex1.3) wit is kun je niet zo makkelijk zien met het blote oog. Daarvoor moet je het schild van de kever onder een microscoop onderzoeken om kleine structuren te kunnen zien. Naast een technologie om het schild te bestuderen heb je andere technologie nodig om de structuur van het schild na te bootsen. Vragen die eenvoudig lijken, hebben vaak een uitgebreid antwoord hebben. Wat is bijvoorbeeld de 'kleur' wit? Dit soort vragen zul je in de loop van deze module proberen te beantwoorden.

Je zult in de module voortdurend gebruik maken van natuurkundige begrippen en principes. Een voorbeeld daarvan, wat je in het volgende hoofdstuk ook zal tegenkomen, is het begrip elektrisch veld. Daarom kijk je in de volgende paragraaf wat dieper naar het voorbeeld van de bij uit [Opdracht 1.3](ex1.3).

<!-- 
\end{verbatim}

Commented {[}LK7{]}: Bron: https://pxhere.com/en/photo/671272 CCO 1.0

\begin{Shaded}
\begin{Highlighting}[]

\NormalTok{\textless{}!{-}{-} ![](https://cdn.mathpix.com/cropped/2024\_06\_24\_0f091e854944f2a35adag{-}06.jpg?height=37\&width=286\&top\_left\_y=620\&top\_left\_x=1539)}
\NormalTok{Commented [LK8]: lets gerichtere vraag, maar ik ben no}
\NormalTok{niet tevreden. De vraag is: wat willen we dat leerlingen toelichten?}

\NormalTok{Commented [LK9]: Ik denk dat het bruggetje meer is: We gaan biologische systemen op microscopische schaal gebruiken als inspiratie voor technologische toepassingen. Daarvoor is nieuwe natuurkunde nodig: de quantummechanica. Vliegen was alleen een voorbeeld voor biomimetica. Het voorbeeld van de kever en het water is een voorbeeld van een microsopische toepassing.[\^{}0] {-}{-}\textgreater{}}

\NormalTok{(E{-}veld)=}
\NormalTok{\#\# Het elektrisch veld}

\NormalTok{Lees onderstaand artikel.}

\NormalTok{\textgreater{} Van al het water op aarde is slechts $2,5 \textbackslash{}\%$ zoet water dat je zou kunnen drinken. Maar we gebruiken dat water ook voor heel veel andere toepassingen. Zo wordt in de Verenigde Staten maar liefst $39 \textbackslash{}\%$ van al het zoete water gebruikt voor het koelen van energiecentrales. Dat water ontsnapt via grote koeltorens naar de atmosfeer (zie [zie figuur 1.3](fig1.3)).}
\NormalTok{\textgreater{} Van al het water op aarde is slechts $2,5 \textbackslash{}\%$ zoet water dat je zou kunnen drinken. Maar we gebruiken dat water ook voor heel veel andere toepassingen. Zo wordt in de Verenigde Staten maar liefst $39 \textbackslash{}\%$ van al het zoete water gebruikt voor het koelen van energiecentrales. Dat water ontsnapt via grote koeltorens naar de atmosfeer ([zie figuur 1.3](fig1.3)).}

\NormalTok{(fig1.3)=}
\NormalTok{(fig1.4)=}
\NormalTok{\textless{}div style="display: flex; justify{-}content: center; flex{-}wrap: wrap; gap: 20px; margin{-}top: 20px;"\textgreater{}}
\NormalTok{    \textless{}figure style="text{-}align: center; margin{-}right: 20px;"\textgreater{}}
\NormalTok{        \textless{}img src="https://cdn.mathpix.com/cropped/2024\_06\_24\_0f091e854944f2a35adag{-}07.jpg?height=383\&width=508\&top\_left\_y=1005\&top\_left\_x=203" alt="Koeltorens" style="height: 200px;"\textgreater{}}
\NormalTok{        \textless{}figcaption\textgreater{}Figuur 1.3: Water ontsnapt via koeltorens naar de atmosfeer.\textless{}/figcaption\textgreater{}}
\NormalTok{    \textless{}/figure\textgreater{}}
\NormalTok{    \textless{}figure\textgreater{}}
\NormalTok{        \textless{}img src="https://cdn.mathpix.com/cropped/2024\_06\_24\_0f091e854944f2a35adag{-}07.jpg?height=383\&width=568\&top\_left\_y=1005\&top\_left\_x=775" alt="Ingenieurs MIT" style="height: 200px;"\textgreater{}}
\NormalTok{        \textless{}figcaption\textgreater{}Figuur 1.4: Ingenieurs van MIT bij hun prototype. Bron: https://energy.mit.edu/news/new{-}systemrecovers{-}fresh{-}water{-}from{-}power{-}plants/\textless{}/figcaption\textgreater{}}
\NormalTok{    \textless{}/figure\textgreater{}}
\NormalTok{\textless{}/div\textgreater{}}

\NormalTok{\textgreater{} Daarom ontwikkelen ingenieurs van het Massachusetts Institute of Technology (MIT) een systeem dat water van koeltorens opvangt om het te kunnen hergebruiken ([figuur 1.4](fig1.4)). Ze spannen daarvoor een elektrisch geladen net boven de koeltoren. Het opstijgende water krijgt de tegenovergestelde lading en wordt daardoor naar het net getrokken. Het water verzamelt zich op het net, waarna het naar beneden drupt. Bijen kunnen op een vergelijkbare manier bloemen vinden en het stuifmeel van de bloem blijft op eenzelfde manier aan de poten van de bij plakken.}

\NormalTok{\textgreater{}Naar: New systems recovers fresh water from power plants, MIT News,}
\NormalTok{https://news.mit.edu/2018/new{-}system{-}recovers{-}fresh{-}water{-}power{-}plants{-}0608}

\NormalTok{Elektrisch geladen deeltjes, zoals het water en het net, hoeven elkaar niet aan te raken om elkaar te kunnen beïnvloeden. Dat komt omdat ieder elektrisch geladen deeltje een elektrisch veld zich heen heeft. Dat veld kun je niet zien met het blote oog of onder een microscoop. Maar je kunt wel voelen dat het er is!}

\NormalTok{\textless{}!{-}{-} Commented [LK11]: Ik vind de uitwerking van $7 \textbackslash{}mathrm\{H\}$ hieronder nog niet zo duidelijk terug komen. Het gaat in de toepassing meer om de aantrekkingskracht en het blijven plakken van stuifmeel. In de uitwerking hier onder meer over het waarnemen van het elektrisch veld. Nog geen suggestie hoe het anders te doen.}

\NormalTok{Commented [LK12]: Dit is een beetje een sprong.}

\NormalTok{Commented [RO13R12]: Ja dat klopt. Komt omdat dit er later bij is geschreven. Ik doe een poging het wat te soepele te maken.}

\NormalTok{Commented [LK14]: Bron}

\NormalTok{Commented [LK15]: Het nieuwsartikel heeft het over een ionenbundel die de druppels eerst een elektrische lading geeft. {-}{-}\textgreater{}}

\NormalTok{\textless{}!{-}{-} \textasciigrave{}\textasciigrave{}\textasciigrave{}}
\NormalTok{Commented [LK16]: Mooi voorbeeld. Dit is we}
\NormalTok{biomimetica na de daad, want WiFi is toch niet geïnspireerc}
\NormalTok{op communicatie tussen bijen?}
\NormalTok{Commented [RO17R16]: Er moet hier een kleine}
\NormalTok{conclusie bijgeschreven worden over het opvangen van}
\NormalTok{water. Het is en uitwerking van 7H uit opdracht 1.2. Zie}
\NormalTok{filmpje https://youtu.be/2pU5Yksk{-}pc}
\NormalTok{Wifi is inderdaad een parallel die je de leerling mee zou}
\NormalTok{kunnen geven maar het is geen biomimetica.}
\NormalTok{\textasciigrave{}\textasciigrave{}\textasciigrave{} {-}{-}\textgreater{}}

\NormalTok{(ex1.5)=}
\NormalTok{\textasciigrave{}\textasciigrave{}\textasciigrave{}\{exercise\} Practicum: statische elektriciteit waarnemen}
\NormalTok{Voor de volgende proef heb je nodig:}
\NormalTok{{-}   kunststof kleding of kunststof doek (bijvoorbeeld van polyester)}
\NormalTok{{-}   ballon}
\NormalTok{Voer de volgende proef uit en beantwoord de vragen:}
\NormalTok{1)  Blaas de ballon op en knoop hem dicht. }
\NormalTok{2)  Wrijf met de ballon een aantal keer snel over je kleding of de doek. De ballon krijgt zo een negatieve elektrische lading.}
\NormalTok{3)  Houd de ballon vervolgens dicht bij haartjes op je arm. }
\NormalTok{4) Beschrijf wat je ziet en beschrijf wat je voelt.}
\NormalTok{5) Bepaal vanaf welke afstand je dit voelt en druk je antwoord uit in je lichaamslengte. Bijvoorbeeld: als je het effect voelt op $0,8 \textbackslash{}mathrm\{\textasciitilde{}m\}$ en jij bent $1,6 \textbackslash{}mathrm\{\textasciitilde{}m\}$ lang, dan voel je het effect op 0,5 lichaamslengtes.}
\end{Highlighting}
\end{Shaded}

De ballon in \href{ex1.5}{Opdracht 1.5} is geladen en oefent op afstand
een kracht uit op de haartjes op je arm. Zo zou je op iedere positie
rond de buis een bepaald effect kunnen verwachten. Natuurkundigen vatten
die invloed afhankelijk van positie samen met het begrip veld. En omdat
het om een elektrische kracht gaat heet het in dit geval een elektrisch
veld.

Het blijkt dat bijen op een vergelijkbare manier het elektrisch veld van
een bloem kunnen voelen. Komen ze dichtbij een bloem, ongeveer vanaf
\(10 \mathrm{~cm}\), dan buigen haartjes op hun lijf af. Dat afbuigen
zorgt voor een signaal in het zenuwstelsel van de bij. Een afstand van
10 centimeter klinkt misschien als weinig maar bedenk dat dit
overeenkomt met ongeveer 5 lichaamslengtes van een bij! Het elektrisch
veld dat een bloem maakt verschilt per bloemsoort. Bijen kunnen dus
kleur, geur en het elektrisch veld gebruiken om verschillende bloemen te
herkennen. In \href{fig1.5}{figuur 1.5} is het elektrisch veld rond een
bloem zichtbaar gemaakt.

(fig1.5)=

\begin{verbatim}
<figure style="text-align: center; margin-right: 20px;">
    <img src="https://cdn.mathpix.com/cropped/2024_06_24_0f091e854944f2a35adag-08.jpg?height=437&width=617&top_left_y=1181&top_left_x=180" alt="Simulation electric field" style="height: 200px;">
    <figcaption>Figuur 1.5: Simulatie van de sterkte van het elektrisch veld tussen een bloem en een bij.</figcaption>
</figure>
\end{verbatim}

Niet alleen tijdens het vliegen maar ook met andere bewegingen maken
bijen steeds wisselende elektrische velden. Ook die wisselende
elektrische velden kunnen bijen waarnemen, zowel met de haartjes als met
een zintuig in hun antennes. Het vermoeden bestaat dat bijen op afstand
met elkaar kunnen communiceren door verschillende elektrische signalen
te herkennen.

Mensen gebruiken ook elektrische (en elektromagnetische) signalen om met
elkaar te communiceren. Denk bijvoorbeeld aan radio en wifi. Dat is niet
direct afgekeken van de natuur maar wel een voorbeeld van hoe de natuur
gebruik maakt van dezelfde natuurkundige principes als de mens.

\hypertarget{sensoren}{%
\subsection{Sensoren}\label{sensoren}}

Uit \href{ex1.5}{Opdracht 1.5} blijkt dat mensen wel het elektrisch veld
kunnen voelen, maar er niet zo gevoelig voor zijn als een bij. De mens
mist hiervoor een fijn ontwikkeld zintuig. Als je een elektrisch veld
wil waarnemen met dezelfde (of meer) precisie als een bij dan moet je
hiervoor een kunstmatig zintuig gebruiken. Zo'n kunstmatig zintuig
noemen we een sensor. Een sensor zet een meetbare grootheid om in een
elektrisch signaal dat we kunnen aflezen op een voltmeter of verwerkt
kan worden door een computer.

```\{exercise\} Zintuigen en sensoren

De mens heeft verschillende zintuigen die je kunt zien als biologische
sensor.

\begin{enumerate}
\def\labelenumi{\alph{enumi})}
\item
  Maak een lijst met zintuigen die een mens heeft.
\item
  Probeer bij elk zintuig een sensor (kunstmatig zintuig) te vinden en
  noteer deze. Bijvoorbeeld: Gehoor ( \(20 \mathrm{~Hz}\) tot
  \(20 \mathrm{kHz}\) ) \(\rightarrow\) Microfoon Als je informatie
  opzoekt vermeld dan de bronnen die je gebruikt hebt.
\item
  Maak ook een lijst met zintuigen die (sommige) dieren wel hebben maar
  mensen niet. Zoek ook daarbij weer een voorbeeld van een sensor en
  vermeld je bronnen.
\end{enumerate}

\begin{verbatim}

(ex1.7)=
```{exercise} Practicum: elektrisch veld sensor
(fig1.6)=
<div style="display: flex; justify-content: center; flex-wrap: wrap; gap: 20px; margin-top: 20px;">
    <figure style="text-align: center; margin-right: 20px;">
        <img src="https://cdn.mathpix.com/cropped/2024_06_24_0f091e854944f2a35adag-10.jpg?height=297&width=561&top_left_y=577&top_left_x=790" alt="circuit" style="height: 200px;">
        <figcaption>Figuur 1.6: Een eenvoudige sensor om het elektrisch veld te meten.</figcaption>
    </figure>
</div>

In deze opdracht maak je met slechts een paar onderdelen een eenvoudige sensor die elektrische velden kan oppikken. Het doel van deze opdracht is iets te bouwen dat werkt maar niet om precies te begrijpen hoe het werkt, dat komt later in de module aan bod. Het schema is gegeven in [figuur 1.6](fig1.6). De onderdelen uit het schema vind je in onderstaande tabel met een foto.



| Onderdeel | Functie | Onderdeel | Functie |
| :--- | :--- | :--- | :--- |
| ![transistor]() | JFET Transistor: regelt de <br> stroom in het circuit. Dit is de eigenlijke sensor. |![batterij]() | Levert energie om de led <br> te laten branden |  |
| ![led]() | led: is een indicator voor <br> de hoeveelheid <br> negatieve lading. Als er <br> geen lading in de buurt <br> is dan brandt de led. De <br> lange pin is de plus (+) | ![batterijclip]() | Batterijclip: om de batterij <br> aan te sluiten. |
| ![weerstand]() | Weerstand: werkt als antenne en beschermt de transistor tegen statische elektriciteit. | ![breadboard]() | Breadboard: Wordt gebruikt om de onderdelen in te klikken en met elkaar te verbinden.|

Voer de volgende opdrachten uit:

a) Teken op het werkblad hoe je de componenten op het breadboard wil prikken zodat ze het schema van [figuur 1.6](fig1.6) vormen. Laat je ontwerp goedkeuren door de docent.

b) Bouw vervolgens jouw opstelling en controleer dat de led gaat branden. Zo niet, dan heb je of een vergissing gemaakt (draai de led eens om) of er is een negatief elektrisch veld in de buurt. Probeer ook het losse uiteinde van de weerstand even aan te raken. Merk op dat het uiteinde van de weerstand los is: dat is geen fout, maar de bedoeling.

c) Onderzoek hoe de sensor reageert in de buurt van een negatief geladen ballon (zie [Opdracht 1.5](ex1.5)).

d) Wapper de ballon heen en weer op verschillende afstanden. Over welke afstand kan de sensor de aanwezigheid van de ballon nog meten? Kun je de sensor van een andere groep beïnvloeden?

e) Werk samen met andere groepen: plaats jullie sensor in een zelfbedacht patroon, bijvoorbeeld allemaal op een rij op gelijke afstand van elkaar. Onderzoek hoe je zo kunt meten hoe het elektrisch veld rond de ballon eruit ziet. Dit is een beetje te vergelijken met de simulatie uit [figuur 1.5](fig1.5) van de bij en de bloem.
\end{verbatim}

\hypertarget{quantummechanica}{%
\subsection{Quantummechanica}\label{quantummechanica}}

Je hebt in \href{ex1.7}{Opdracht 1.7} je eerste quantumsensor gemaakt!
De led en de transistor zijn namelijk gemaakt van het
halfgeleidermateriaal silicium. Zoals je hebt gezien kun je prima
apparaten bouwen met halfgeleiders zonder precies te weten waarom het
werkt. Dat is wat elektrotechnisch ingenieurs doen: met een paar
vuistregels ontwerpen ze heel veel nuttige circuits. Maar wil je beter
begrijpen wat er in de halfgeleider gebeurt en waarom het reageert op
een elektrisch veld, dan moet je een beetje quantummechanica kennen.

```\{exercise\} Wat weet je al van quantum?

Je hebt de term quantum, met of zonder `mechanica', misschien al eens
gehoord. Bespreek met elkaar of en waar je de term bent tegengekomen.
Noteer begrippen die volgens jullie met quantum te maken hebben en wat
`quantum' volgens jullie betekent. ```

In het volgende hoofdstuk duik je dieper in de wereld van de
quantummechanica.
